In the last few years, Internet of Things (IoT) has grown into a big part of both people's everyday life and especially industry. IoT as a term is used for devices and other things that are connected and communicate via the internet. The market for Internet of Things solutions and services is projected to grow at 26.4\% Compound Annual Growth Rate until the year 2029 per a report from the Fortune Business Insights and the growth forecast looks to follow an exponential trend. Biggest shares by end use industry have been healthcare and manufacturing in the year 2021. \cite{Insights2022a}

However, just the data alone from devices is not that valuable if it is not processed or analyzed in any way. Some people even go as far to say that "The data generated from IoT devices turns out to be of value only if it gets subjected to analysis" \cite{Joseph}, although in some cases alarms can be set up with simple thresholds only. Still, the role of analytics with data from IoT devices is undeniable. The technologies regarding different data analytics solutions are becoming more and more sophisticated and that combined with the steady ascent of computing resources, different data analytics solutions are used more and more. As reported by Fortune Business Insights, the global data science platform market is projected to grow at a Compound Annual Growth Rate of 29.0\% \cite{Insights2022}.

The client \todo{"toimeksiantaja"} of this thesis is Wapice Ltd. Wapice is a Finnish technology company that is specialized in industrial software solutions. One of Wapice's biggest products is the IoT-TICKET, with which the customers can remotely control and monitor their Internet of Things devices and data from them. The IoT-TICKET has a web-based user interface and it can be used for different kind of applications with a low-code development platform. IoT-TICKET will be introduced more in depth in the chapter 2.

During the last year, Wapice has been building a solution that would enable more in-depth analytics for customers using the IoT-TICKET version 4. This product has been named the Data Analytics Suite (DAS) version 2. The Data Analytics Suite will be used as an add-on with IoT-TICKET. As customers that use the IoT-TICKET already have the data from their devices and some logic, alarms ett cetera in IoT-TICKET, accessing it easily for deeper analytics creates more value for their data as for example the logic can be expanded. The Data Analytics Suite aims to tackle the exact problems presented in paragraph two and is placed in the field of both data analytics and iot services. This thesis revolves around the DAS project and the software development kit (SDK) that has been developed as the user interface for the DAS.

Currently, the Data Analytics Suite uses Azure's DataBricks as the analytics platform that is linked with IoT-TICKET. It is described by Azure as "a unified set of tools for building, deploying, sharing, and maintaining enterprise-grade data solutions at scale." \cite{Azure2022} In the future, other platforms will be considered along with DataBricks to adapt different customers, if they already use another platform for their analysis, or for some other reason want to use another service.

As the DAS project is already in an advanced state and the basic functionalities are implemented into the SDK, this thesis will take a look on whether it has met its requirements concerning the already implemented functionalities and the practices used with the structure of the project. As a result, it is desired to get an objective review on what aspects have been succeeded in, and how the package could be made better. The research questions in this thesis are: "How to analyze data harvested with IoT-TICKET in a way that is suits the data analyst's needs?" and "How to build the das-sdk modularily and in a clear way for future development?". The first of these questions aims to take account into the usability and functionality aspects while the latter takes a look under the hood of the package and answers whether the project is build with good practices.

In addition to an introduction to IoT and IoT-TICKET, DataBricks and the DAS project are introduced in chapter 2. Also, user experience and usability are defined as terms. In chapter 3, the methods for analyzing user experience, usability and the project structure are examined. In this section of the thesis, it is also defined, what kind of items the wanted results include. The results will be obtained with mostly qualitative analysis. In chapter 4, the analysis is done for the research questions defined above. In chapter 5, the results are analyzed. Finally, in chapter 6, the thesis and its results are summarized.
