%%%%%%%%%%%%%%%%%%%%%%%%%%%%%%%%%%%%%%%%%%%%%%%%%%%%%%%%%%%
%% Congratulations, you've made an excellent choice
%% of writing your Tampere University thesis using
%% the LaTeX system. This document attempts to be
%% as complete a template as possible to let you focus
%% in the most important part: the writing itself.
%% Thus the details regarding the visual appearance
%% and even structure have already been worked out
%% for you!
%%
%% I sincerely hope you will find this template useful
%% in completing your thesis project. I've tried to
%% add comments (followed by the % sign) to clarify
%% the structure and purpose of some of the commands.
%% Most of the magic happens in the file ./tauthesis,
%% which you are more than welcome to take a look at.
%% Just refrain from editing it in the most crucial
%% versions of the thesis!
%%
%% I wish you and your thesis project the best of luck!
%% If you've got any suggestions for improving this template,
%% please contact me via email at
%%
%% ville.koljonen (at) tuni.fi.
%%
%% Yours,
%%
%% Ville Koljonen
%% 16th May 2019
%%
%% PS. This template or its associated class file don't
%% come with a warranty. The content is provided as-is,
%% without even the implied promise of fitness to the
%% mentioned purpose. You, as the author of the thesis,
%% are responsible for the entire work, including the
%% provided material. No one else is liable to you for
%% any damage inflicted on you or your thesis were it
%% caused by using this template or not.
%%%%%%%%%%%%%%%%%%%%%%%%%%%%%%%%%%%%%%%%%%%%%%%%%%%%%%%%%%%

%%%%% PREAMBLE %%%%%

%%%%% Document class declaration.
% The possible optional arguments are
%finnish - thesis in Finnish (default)
%english - thesis in English
%numeric - citations in numeric style (default)
%authoryear - citations in author-year style
%draft - for faster non-final works, also skips images
%           (recommended, remove in the final version)
%programs - if you wish to display code snippets
% Example: \documentclass[english, authoryear]{tauthesis}
%          thesis in English with author-year citations
\documentclass[finnish, numeric]{tauthesis}  % draft -> kuvat ei lataudu

% The glossaries package throws a warning:
% No language module detected for 'Finnish'.
% You can safely ignore this. All other
% warnings should be taken care of!

%%%%% Your packages.
% Before adding packages, see if they can be found
% in ./tauthesis already. If you're not sure that
% you need a certain package, don't include it in
% the document! This can dramatically reduce
% compilation time.

\usepackage{todonotes}
% Add [disable]{todonotes} to remove todonotes and remove the setlength'

\usepackage{hyphenat}

\usepackage[final]{listings}
\usepackage{inconsolata}

\renewcommand{\lstlistingname}{Ohjelma}

\usepackage{graphicx}
\usepackage{pdfpages}

% Graphs
% \usepackage{pgfplots}
% \pgfplotsset{compat=1.15}

% Subfigures and wrapping text
% \usepackage{subcaption}

% Mathematics packages
\usepackage{amsmath, amssymb, amsthm}
%\usepackage{bm}

% The chemistry packages
% \usepackage{chemfig}
% \usepackage[version=4]{mhchem}

% Text hyperlinking
% \usepackage{hyperref}
% \hypersetup{hidelinks}

% (SI) unit handling
\usepackage{siunitx}

\sisetup{
    detect-all,
    math-sf=\mathrm,
    exponent-product=\cdot,
    output-decimal-marker={.} % for theses not in FINNISH!
}

%%%%% Your commands.

% Print verbatim LaTeX commands
\newcommand{\verbcommand}[1]{\texttt{\textbackslash #1}}


% Basic theorems in Finnish and in English.
% Remove [chapter] if you wish a simply
% running enumeration.
% \newtheorem{lause}{Lause}[chapter]
% \newtheorem{theorem}[lause]{Theorem}

% \newtheorem{apulause}[lause]{Apulause}
% \newtheorem{lemma}[lause]{Lemma}

% Use these versions for individually
% enumerated lemmas
% \newtheorem{apulause}{Apulause}[chapter]
% \newtheorem{lemma}{Lemma}[chapter]

% Definition style
% \theoremstyle{definition}
% \newtheorem{maaritelma}{Määritelmä}[chapter]
% \newtheorem{definition}[maaritelma]{Definition}
% examples in this style

%%%%% Glossary information.

\loadglsentries[main]{tex/sanasto.tex}
\makeglossaries

%%%%% in citation information.

\addbibresource{tex/library.bib}
\emergencystretch=2em

\begin{document}

%%%%% FRONT MATTER %%%%%

\frontmatter

%%%%% Thesis information and title page.

% The titles of the work. If there is no subtitle,
% leave the arguments empty. Pass the title in
% the primary language as the first argument
% and its translation to the secondary language
% as the second.
\title{Pääotsikko}{pääotsikko englanniksi}
\subtitle{Alaotsikko}{Alaotsikko englanniksi}

% The author name.
\author{Ossi Koski}

% The examiner information.
% If your work has multiple examiners, replace with
% \examiner[<label>]{<name> \\ <name>}
% where <label> is an appropriate (plural) label,
% e.g. Examiners or Tarkastajat, and <name>s are
% replaced by the examiner names, each on their
% separate line.
\examiner[Tarkastaja]{Tarkastajan nimi}

% The finishing date of the thesis (YYYY-MM-DD).
\finishdate{2022}{10}{24}

% The type of the thesis (e.g. Kandidaatintyö
% or Master of Science Thesis) in the primary
% and the secondary languages of the thesis.
\thesistype{Diplomityö}{Master of Science Thesis}

% The faculty and degree programme names in
% the primary and the secondary languages
% of the thesis.
\facultyname{Tekniikan ja luonnontieteiden tiedekunta}{Faculty of Engineering and Natural Sciences}
\programmename{Teknisten tieteiden kandidaatin tutkinto-ohjelma}{Bachelor's Programme in Engineering Sciences}

% The keywords to the thesis in the primary
% and the secondary languages of the thesis.
\keywords%
    {avainsanalista, avainsana}

\maketitle

% Write the abstract(s) and the preface
% into a separate file for the sake of clarity.
% Pass the appropriate file name as the first
% argument to these commands. Put the \abstract
% in the primary language first and the
% \otherabstract in the secondary language second.
% Those who do not speak Finnish only need the
% first abstract. The second argument of
% the \preface command takes the place where
% the thesis was signed in.
\abstract{tex/tiivistelma.tex}
\otherabstract{tex/abstract.tex}
\preface{tex/alkusanat.tex}{Tampereella}

%%%%% Table of contents.

\tableofcontents

%%%%% Lists of figures, tables, listings and terms.

% Print the lists of figures and/or tables.
% Uncomment either of these commands as required.
% Both are optional, but if there are many important
% figures/tables, listing them may be a good idea.

% \listoffigures
% \listoftables
% \lstlistoflistings

% Print the glossary of terms.

\glossary

%%%%% MAIN MATTER %%%%%

\mainmatter

% Write each of the chapters of the thesis
% into a separate file for the sake of clarity.
% They can be \input as shown below. Give both
% the chapters and their files as descriptive
% names as possible.

\chapter{Johdanto}
\label{ch:johdanto}
Johdanto


\chapter{Yhteenveto}
\label{ch:yhteenveto}
Yhteenveto



%%%%% Bibliography/references.

% Print the bibliography according to the
% information in ./tex/references.bib
% and the in-line citations used in the body
% of the thesis.
% \emergencystretch=2em
\printbibliography[heading=bibintoc]

%%%%% Appendices.

% Use only if it clarifies the structure of
% the document. Remember to introduce each
% appendix.

%\begin{appendices}
%\end{appendices}

\end{document}
